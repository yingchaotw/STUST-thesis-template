%----------------------------------------------------------------------------------------
%	PACKAGES AND OTHER DOCUMENT CONFIGURATIONS
%----------------------------------------------------------------------------------------

\documentclass[oneside,
12pt, % The default document font size, options: 10pt, 11pt, 12pt
english, % ngerman for German
onehalfspacing, % Single line spacing, alternatives: onehalfspacing or doublespacing
]{../Configurations/STUST-MS-Thesis} % The class file specifying the document structure

\setcounter{secnumdepth}{6}
\setcounter{tocdepth}{6}

%----------------------------------------------------------------------------------------
% 字體設定
%----------------------------------------------------------------------------------------
\def\fontpath{Path=\string../fonts/}		% 設定字體目錄
\usepackage[BoldFont,SlantFont]{xeCJK}		% xelatex 中文套件設定
\usepackage[LGR,T1]{fontenc} 				% Output font encoding for international characters
\usepackage{palatino} 						% Use the Palatino font by default
\usepackage{lipsum}							% Easy access to the Lorem Ipsum and other dummy texts
\usepackage{fontspec}						% Advanced font selection in XELATEX
\usepackage{babel} 							% 外國語系字母支援
\usepackage{lmodern}						% 大綱格式的拉丁現代字體
\usepackage{anyfontsize}					% Select any font size in LATEX
\usepackage{xltxtra}						% “Extras” for LATEX users of XETEX
\usepackage{type1cm,type1ec} 				% If your installation uses scaleable versions of the Computer Modern or European Computer Modern (EC) fonts
\xeCJKsetup{PunctStyle=kaiming}
\defaultfontfeatures{AutoFakeBold=2.5,AutoFakeSlant=.2}

% 設定英文字體 (Times New Roman)
\setmainfont[\fontpath]{times}
\setsansfont[\fontpath]{times}

% 設定中文字體 (教育部標準楷體)
\setCJKmainfont[\fontpath]{ukai.ttc}					% 設置正文羅馬族的CJK字體,影響\rmfamily和\textrm 的字體
\setCJKmonofont[\fontpath]{edukai-4.0}					% 設置正文等寬族的CJK字體,影響\ttfamily 和 \texttt 的字體
\setCJKsansfont[\fontpath]{edukai-4.0}					% 設置正文無襯線族的CJK字體,影響\sffamily和\textsf 的字體
\setCJKfamilyfont{edukai-4.0}[\fontpath]{edukai-4.0}    % 教育部楷書
\CJKfontspec[\fontpath]{edukai-4.0}
\defaultCJKfontfeatures{AutoFakeBold=2.5,AutoFakeSlant=.2}

\renewcommand*{\rmdefault}{lmr}
\renewcommand*{\sfdefault}{lmss}
\renewcommand*{\ttdefault}{lmtt}
\newcommand*{\edukai}{\CJKfamily{edukai-4.0}}

\newCJKfontfamily\kai{edukai-4.0}[\fontpath,AutoFakeBold]
\newCJKfontfamily\mincho{SawarabiMincho-Regular}[\fontpath,AutoFakeBold]
\newCJKfontfamily\ukai{ukai.ttc}[\fontpath,AutoFakeBold]

%----------------------------------------------------------------------------------------
% 排版套件 / 設定
%----------------------------------------------------------------------------------------
\usepackage{indentfirst} \setlength{\parindent}{2em} 		% 首段文章縮排套件
\usepackage{ulem}											% 底線樣式
\usepackage{microtype}

% 解決 XeTeX 中文的斷行問題
\XeTeXlinebreaklocale "zh"
\XeTeXlinebreakskip = 0pt plus 1pt
\XeTeXlinebreakskip = 0pt plus 1pt minus 0.1pt

%----------------------------------------------------------------------------------------
\usepackage{array}
\usepackage{longtable}
\usepackage{colortbl}			%表格標題註解之巨集套件
\usepackage[final]{pdfpages}    %PDFの各ページを挿入するためのパッケージです
\usepackage{url}         		% URLをリンクとして表示するためのパッケージ
\usepackage{zhnumber}
\linespread{1.5} 				% 行距1.5倍

\usepackage{docmute}
\usepackage{xcolor}
\usepackage{lettrine}  			% used for chinese bigger capital
\usepackage{../Package/slashbox}
\usepackage{booktabs}
\usepackage{multirow}
\usepackage{forest}
\usepackage{ifthen}
\usepackage[strict]{changepage}
\usepackage{lscape} 			% Place selected parts of a document in landscape
\usepackage{dcolumn} 			% Align on the decimal point of numbers in tabular columns
% \usepackage{blindtext}
\usepackage{enumerate}
\usepackage{enumitem}

%----------------------------------------------------------------------------------------
% 參考文獻 Reference package
%----------------------------------------------------------------------------------------
\usepackage[redeflists]{IEEEtrantools}   %複数行にわたった数式を書く場合には,IEEEeqnarray 環境が便利です
\usepackage[hyperref=true,backend=biber,sorting=none,backref=true]{biblatex}
% \usepackage[backend=biber,sorting=none,style=ieee]{biblatex}  %
\addbibresource{\RefPath} % The filename of the bibliography

\frontmatter % Use roman page numbering style (i, ii, iii, iv...) for the pre-content pages
\pagestyle{plain} % Default to the plain heading style until the thesis style is called for the body content
\usepackage{makecell}
\cleardoublepage
\usepackage[autostyle=true]{csquotes} % Required to generate language-dependent quotes in the bibliography
% \usepackage[utf8]{inputenc} % Required for inputting international characters
\usepackage{datetime}
\usepackage{xparse}

% solve the Underfull \hbox (badness 2205) issues
\usepackage{etoolbox}
% Variant A
\apptocmd{\sloppy}{\hbadness 10000\relax}{}{}	
% Variant B
% \apptocmd{\thebibliography}{\raggedright}{}{}

\addbibresource{\RefPath} % The filename of the bibliography

%----------------------------------------------------------------------------------------
% 繪圖函數 Graph functions
%----------------------------------------------------------------------------------------
\usepackage{pgfplots}
\usepgfplotslibrary{colorbrewer}
\pgfplotsset{compat=newest,compat/show suggested version=false}

%----------------------------------------------------------------------------------------
% 圖片套件 Image functions
%----------------------------------------------------------------------------------------
\usepackage{tikz} % Required for drawing custom shapes
\usepackage{eso-pic,picture} % \AddToShipoutPictureBG is esp-pic package's function

%----------------------------------------------------------------------------------------
%	Source Code include Setup
%----------------------------------------------------------------------------------------
\usepackage{listings}
\lstset{
			language={C},   %言語の指定.C言語ならCとします
			basicstyle={\ttfamily}, %標準の書体
			identifierstyle={\small},
			commentstyle={\small\ttfamily \color[rgb]{0,0.5,0}},    %注釈の書体 
			keywordstyle={\small\bfseries \color[rgb]{0,0,1}},      %キーワード(int, ifなど)の書体
			stringstyle={\small\ttfamily \color[rgb]{1,0,1}},
			frame=tRBl, %フレームの指定
			framesep=10pt, %フレームと中身(コード)の間隔
			breaklines=true, %行が長くなったときの自動改行
			linewidth=15cm, %フレームの横幅
			lineskip=-0.5ex, %行間の調整
			columns=[l]{fullflexible},  %書体による列幅の違いを調整するか
			numbers=left,
			stepnumber=1,   %行番号をいくつとばしで表示するか
			numbersep=14pt,
			tabsize=2, %Tabを何文字幅にするかの指定
			morecomment=[l]{//}
		}
